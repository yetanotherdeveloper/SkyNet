%& --translate-file il2-t1
\documentclass[a4paper,10pt]{article}
%\usepackage[polish]{babel}
%\usepackage{polski}
\usepackage{makeidx}
%\usepackage[utf-8]{inputenc}
%\usepackage[T1]{fontenc}
\usepackage[a4paper,left=2.5cm,right=2.0cm,top=1.5cm,bottom=1.5cm]{geometry}
% Title Page
\title{Project SkyNet design notes}
\author{Jacek Czaja}
\makeindex

\begin{document}

\maketitle

\tableofcontents

\section{Introduction}
SkyNet is to be testing platform for various algorithms. Including:
\begin{itemize}
\item Machine Learning (Neural Networks, Support Vector Machines, Perceptions)
\item Artificial Intelligence (A* , Alpha-Beta )
\item Quantum Computation (simulation of algorithms of Quantum computer)
\end{itemize}

Testing is meant to be functional (If generated data match reference data) and performance (time which takes for OpenCL based solution to do its computation)

\section{Architecture Design}
Architecture will have modular design.
Main module will deliver interface for plugins.
Plugins(Modules) will implement interface.
Plugins will be independent to each other.
Interface will be something like:
\begin{verbatim}
class ModuleInterface
{
    RunTest()
    PrintDescription()
};
\end{verbatim}

Main Module at the beginning will load all modules from directory where modules are kept.
How Main Module is to determine what's inside of given module is  yet to be described.

\section{Usage Interface}
SkyNet can be operated from command line. Options:
\begin{itemize}
\item list (lists available modules)
\item help (print info about usage of SkyNet)
\item test (Run Tests from specific module)
\end{itemize}

Starting without parameters will run all possible tests of available modules.

\end{document}

