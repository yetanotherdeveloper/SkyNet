%& --translate-file il2-t1
\documentclass[a4paper,10pt]{article}
%\usepackage[polish]{babel}
%\usepackage{polski}
\usepackage{makeidx}
%\usepackage[utf-8]{inputenc}
%\usepackage[T1]{fontenc}
\usepackage[a4paper,left=2.5cm,right=2.0cm,top=1.5cm,bottom=1.5cm]{geometry}
% Title Page
\title{Project SkyNet design notes}
\author{Jacek Czaja}
\makeindex

\begin{document}

\maketitle

\tableofcontents

\section{Introduction}
SkyNet is to be testing platform for various algorithms. Including:
\begin{itemize}
\item Machine Learning (Neural Networks, Support Vector Machines, Perceptions)
\item Artificial Intelligence (A* , Alpha-Beta )
\item Quantum Computation (simulation of algorithms of Quantum computer)
\end{itemize}

Testing is meant to be functional (If generated data match reference data) and performance (time which takes for OpenCL based solution to do its computation)

\section{Architecture Design}
Architecture will have modular design.
Main module will deliver interface for plugins.
Plugins(Modules) will implement interface.
Plugins will be independent to each other.
Interface will be something like:
\begin{verbatim}
class ModuleInterface
{
    RunTest()
    PrintDescription()
};
\end{verbatim}

Main Module at the beginning will load all modules from directory where modules are kept.
How Main Module is to determine what's inside of given module is  yet to be described.

SkyNet holds lists of diffrent types of modules eg. classifiers (PLA, SVM)

Tests for diffrent devices to be run on separate threads

Test modules will conduct/implement testing scenarios like. randomClassification etc. Module will return training samples,
performa validation, verification 

So, module will be given data and commandqueue to send kernel to, context and device that compilation is taking place against

There will be a module handling OpenCL :
Implementing SkyNetOpenCLHelper module , which is opencl interacting routines to creat context, buildp rograms, kernels etc. and handling errors


Charts module. Module to generate , dump data. Can also draw gnuplot script.
Scheme of working:
\begin{enumerate}
\item Each module may have its instance
\item Constructor to create directory: module about string + PID
\item We put history of flags and for each weights that will be dumped and charts can be generated for them
\item Generation of movie by script is also to be done
\end{enumerate}

compilation flags for building program are taken from device version passed to OpenCL Helper module




\begin{itemize}
\item creation of command queue
\item program building/creating
\item kernel creation
\item handling or errors
\end{itemize}

There are modules delivering tests to be done and validating results.
For example randomPointsClassification module. Its purpose is to:
\begin{itemize}
\item generated random input
\item initalize weights
\item validate generated weights eg. check if weights do well on testing example
\item test generalization of selected hypothesis 
\item Training Points can also be stored
\item test module also to have Diagnostic class instance so to have picture of generalization
\end{itemize}

Where to deliver reference CPU implementation? I think good place for CPU reference implementation is module itself.
So Protocol will be extended with method RunReferencePLA

Classification modules when requested will return reference to classification data of last processed set. Eg.

consider unique pointer instead normal one
Implement charts generation (Error how does it change) 

- Implement user interface 
\section{Usage Interface}
SkyNet can be operated from command line. Options:
\begin{itemize}
\item list (lists available modules)
\item help (print info about usage of SkyNet)
\item test (Run Tests from specific module)
\end{itemize}

Starting without parameters will run all possible tests of available modules.


\section{Installation}
\subsection{Building with GCC for generation of clang\_complete\_config}
\begin{verbatim}
CXX='/home/jczaja/.vim/bundle/clang_complete/bin/cc_args.py g++'   cmake ../code/
make VERBOSE=1
make copy_clang_complete
\end{verbatim}
\subsection{Building with Clang}
\begin{verbatim}
CXX='/home/jczaja/.vim/bundle/clang_complete/bin/cc_args.py clang++'   cmake ../code/
make VERBOSE=1
make copy_clang_complete
\end{verbatim}


\section{TODO}
\begin{itemize}
\item Make exceptions,errors handling for SKyNetOpenCLHelper module
\item Compilation flags for kernels should come from SkyNet module alond with device send or generate flags itself in helper
\item make common module handling init of opencl and kernel/program building
\item Make a git branches: master and milestone(release)
\item make/lear proper c++ exceptions usage to handle errors of OpenCL initialization
\item Write a report that clang complete works badly with C++ openCL
\item each module should provide also CPU plain implementation of algorithm
\item Modules are algorithms and Devices( hardware layer) are to be initialized in SkyNet core
\item Command Line parser (use gnuopt)
\item Create Contexts assigned to given devices/platforms
\item Mapping of errors(ints) to strings
\item As for drawing animation out generated weights. We could make a global output buffer
containing weights as they are being updated
\item Create a module creating gnuplot charts/ generting scripts making a movie out of generated charts
\item User Interface (command line)
\item serialization / loading initial weights from file 
\item make gnuplot script generation of charts of learning steps , and make video out of it
\item coding standard fixes:
\begin{enumerate}
\item empty functions , braces in the same line
\item force space removal in args of function call
\item brace after for loop is not moved to new line
\end{enumerate}
\end{itemize}

\end{document}

